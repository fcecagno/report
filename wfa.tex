% THIS IS SIGPROC-SP.TEX - VERSION 3.1
% WORKS WITH V3.2SP OF ACM_PROC_ARTICLE-SP.CLS
% APRIL 2009
%
% It is an example file showing how to use the 'acm_proc_article-sp.cls' V3.2SP
% LaTeX2e document class file for Conference Proceedings submissions.
% ----------------------------------------------------------------------------------------------------------------
% This .tex file (and associated .cls V3.2SP) *DOES NOT* produce:
%       1) The Permission Statement
%       2) The Conference (location) Info information
%       3) The Copyright Line with ACM data
%       4) Page numbering
% ---------------------------------------------------------------------------------------------------------------
% It is an example which *does* use the .bib file (from which the .bbl file
% is produced).
% REMEMBER HOWEVER: After having produced the .bbl file,
% and prior to final submission,
% you need to 'insert'  your .bbl file into your source .tex file so as to provide
% ONE 'self-contained' source file.
%
% Questions regarding SIGS should be sent to
% Adrienne Griscti ---> griscti@acm.org
%
% Questions/suggestions regarding the guidelines, .tex and .cls files, etc. to
% Gerald Murray ---> murray@hq.acm.org
%
% For tracking purposes - this is V3.1SP - APRIL 2009

\documentclass{acm_proc_article-sp}

\usepackage[utf8]{inputenc}
\usepackage[brazilian]{babel}
\usepackage{color}
\usepackage{listings}
\usepackage{textcomp}
%\usepackage{algorithm2e}
%\hyphenation{po-de-ría-mos}
\newcommand{\todo}[1]{\textcolor[rgb]{1.00,0.00,0.00}{\bf \uppercase{#1}}}

\begin{document}

\title{Mconf-Mobile: videoconferência \\BigBlueButton no Android}
%
% You need the command \numberofauthors to handle the 'placement
% and alignment' of the authors beneath the title.
%
% For aesthetic reasons, we recommend 'three authors at a time'
% i.e. three 'name/affiliation blocks' be placed beneath the title.
%
% NOTE: You are NOT restricted in how many 'rows' of
% "name/affiliations" may appear. We just ask that you restrict
% the number of 'columns' to three.
%
% Because of the available 'opening page real-estate'
% we ask you to refrain from putting more than six authors
% (two rows with three columns) beneath the article title.
% More than six makes the first-page appear very cluttered indeed.
%
% Use the \alignauthor commands to handle the names
% and affiliations for an 'aesthetic maximum' of six authors.
% Add names, affiliations, addresses for
% the seventh etc. author(s) as the argument for the
% \additionalauthors command.
% These 'additional authors' will be output/set for you
% without further effort on your part as the last section in
% the body of your article BEFORE References or any Appendices.

\numberofauthors{1} %  in this sample file, there are a *total*
% of EIGHT authors. SIX appear on the 'first-page' (for formatting
% reasons) and the remaining two appear in the \additionalauthors section.
%
\author{
% You can go ahead and credit any number of authors here,
% e.g. one 'row of three' or two rows (consisting of one row of three
% and a second row of one, two or three).
%
% The command \alignauthor (no curly braces needed) should
% precede each author name, affiliation/snail-mail address and
% e-mail address. Additionally, tag each line of
% affiliation/address with \affaddr, and tag the
% e-mail address with \email.
%
% 1st. author
%\alignauthor
Felipe Cecagno,
Valter Roesler\\
       \affaddr{Instituto de Informática}\\
       \affaddr{Universidade Federal do Rio Grande do Sul}\\
       \affaddr{Porto Alegre, Rio Grande do Sul, Brasil}\\
       \email{\{fcecagno, roesler\}@inf.ufrgs.br}
}
\date{10 de julho de 2011}

\maketitle
\begin{resumo}
  Este artigo tem o objetivo de apresentar o resultado parcial do projeto GT-Mconf. O BBB-Android é um aplicativo para dispositivos móveis com sistema operacional Android, e através dele é possível interagir com outros usuários em uma videoconferência no sistema BigBlueButton. Serão apresentados alguns objetivos do projeto, a arquitetura da solução e suas principais funcionalidades.
\end{resumo}

\begin{abstract}
  This article aims to present the partial result of the GT-Mconf project. The BBB-Android is an application for mobile devices with Android operational system, and \todo{???} it's possible to interact with other users in a videoconferencing on BigBlueButton system. It will be presented some goals of the project, the solution architecture and the main functionalities.

\todo{traduzir abstract}
\end{abstract}

% http://www.acm.org/about/class/ccs98-html
% A category with the (minimum) three required fields
%\category{}{Peer-to-peer multimedia systems and streaming}{}
%\category{}{Software development using multimedia techniques}{}
%\category{H.4}{Information Systems Applications}{Miscellaneous}


\category{C.3}{Special-purpose and application-based systems}{Real-time and embedded systems}
\category{J.7}{Computers in other systems}{Real time}
\category{I.6}{Simulation and modeling}{Applications}
\category{H.4}{Information systems applications}{Communications Applications}{Computer conferencing, teleconferencing, and videoconferencing}

\terms{Design, Experimentation}

\keywords{Interação multimídia de tempo real, Vídeo, JNI}

\section{Introdução}

\todo{falar mais do GT-Mconf}

Dispositivos móveis inteligentes vem ganhando a cada dia mais espaço e popularidade entre as pessoas. A troca de telefones celulares burros por \emph{smartphones} acontece por diversos motivos, e um deles é a grande variedade de aplicativos disponíveis. Dentre eles, os mais populares são os games \todo{referência aqui}, que aproveitam a capacidade de processamento, gráfico ou não, para atrair os usuários.

Além dos jogos, aplicações de vídeo em tempo real vêm crescendo no mercado de aplicativos. Clientes VoIP como o Sipdroid e aplicativos populares por sua versão para desktop como o Skype já incluem a funcionalidade de vídeo chamada nos seus aplicativos para dispositivos móveis. Além destes, o Fring permite chamada em grupos de até quatro pessoas.

Entretanto, tanto o Skype quanto o Fring permitem chamadas apenas para clientes de mesmo tipo, ou seja, Skype só fala com Skype e Fring só fala com Fring. Os clientes VoIP são mais genéricos, mas ao mesmo tempo mais limitados quanto a funcionalidades.

O objetivo do subprojeto Mconf-Mobile é desenvolver um aplicativo para videoconferência integrado ao ambiente BigBlueButton. Com isso visa-se a interação plena entre dispositivos móveis e clientes desktop conectados ao sistema através da Web.

\section{Arquitetura}

O BigBlueButton é um sistema de código aberto para webconferência voltado principalmente para ensino à distância. Ele oferece funcionalidades como interação por áudio, vídeo e bate-papo (público e privado), além de compartilhamento da tela do computador e apresentação síncrona de slides e documentos.

O servidor do BigBlueButton utiliza o Red5, que é uma implementação livre do Adobe Flash Media Server. Por isso, o cliente Web é implementado em Flash, e toda a comunicação com o servidor é feita através do protocolo RTMP. Além da comunicação por vídeo, realizada de forma transparente entre o cliente Flash e o servidor Red5, são utilizados Remote Shared Objects (RSO) e Remote Procedure Calls (RPC) para as demais trocas de mensagens, como bate-papo e gerenciamento de status de participantes.

Para áudio, o sistema utiliza um servidor VoIP, que pode ser o FreeSWITCH ou o Asterisk. No cliente Web, é utilizada uma implementação de telefone SIP em Flash.

O Android oferece suporte a plataforma Flash a partir da versão 2.2, mas optou-se por desenvolver um aplicativo nativo através da SDK padrão para permitir compatibilidade com as versões anteriores. A arquitetura do aplicativo Android será detalhada nas subseções seguintes.

\subsection{Flazr}

Toda a comunicação RTMP entre o aplicativo e o servidor BigBlueButton é feita através da biblioteca Flazr, uma implementação livre em Java de protocolos de streaming multimídia. A biblioteca implementa partes importantes do protocolo RTMP, como por exemplo o \emph{handshake} inicial e multiplexação/demultiplexação de mensagens de controle, áudio e vídeo, e comandos de RPC e controle de fluxos de dados.

Entretanto, a biblioteca não oferece suporte a RSO, recurso necessário para realização plena da comunicação com o servidor. Por exemplo, atualização de status de participantes da conferência como informação de quem entrou e quem saíu, publicação de uma stream de vídeo ou então levantar a mão para pedir permissão para falar são ações informadas através do objeto \textbf{participantsSO}. Outro exemplo é a troca de mensagens de bate-papo públicas, que são realizadas através do objeto \textbf{chatSO}. Por isso tal funcionalidade teve de ser adicionada à biblioteca, e foi utilizada como base a implementação de RSO do próprio Red5.

\subsection{BBB-Java}

A fim de possibilitar o desenvolvimento de novos aplicativos integrados ao BigBlueButton, buscou-se desacoplar ao máximo a biblioteca responsável pela comunicação com o servidor de webconferência do código do aplicativo Android. Por isso foi criado o BBB-Java, que é a biblioteca responsável pela comunicação do cliente com o servidor.

Dentre os recursos oferecidos pela biblioteca estão:
\begin{itemize}
 \item Gerenciamento de salas de conferência (pesquisa das salas existentes, criação de novas salas e acesso a uma sala específica);
 \item Atualização do status dos participantes da sessão;
 \item Troca de mensagens de bate-papo (público e privado);
 \item Recepção e envio de dados de vídeo.
\end{itemize}

O BBB-Java utiliza a versão modificada da biblioteca Flazr para realizar a comunicação RTMP com o servidor. Disponibilizando uma biblioteca de interação com o BigBlueButton espera-se ver crescer em no futuro o número de aplicativos capazes de interagir com o sistema. Além disso, a biblioteca pode ser utilizada para a criação de robôs de teste do sistema, que podem ser úteis para a realização de testes de funcionamento e de carga.

\subsection{BBB-Android}

O BBB-Android é um aplicativo que permite a interação entre usuários em dispositivos móveis dotados do sistema operacional Android e usuários de computadores pessoais. Esse aplicativo foi desenvolvido utilizando o \emph{Software Development Kit} padrão do Android, além do \emph{Native Development Kit} do Android, que possibilita integração de código desenvolvido nas linguagens C e C++.

Considerando o fato de que o servidor BigBlueButton utiliza um servidor VoIP para lidar com os fluxos de áudio, para o BBB-Android adotou-se a estratégia de integrar ao aplicativo uma solução de VoIP de código aberto conhecida e consolidada. Optou-se por realizar a integração do aplicativo Sipdroid, um dos softphones SIP mais populares do Android Market, ao BBB-Android, adicionando assim a funcionalidade de interação por áudio através do aplicativo.

\todo{FFmpeg}

\section{Principais funcionalidades}

\subsection{Bate-papo}

\subsection{Áudio}

O BBB-Android permite que usuários conectados a uma conferência através de um dispositivo móvel interajam de forma natural com os demais participantes através de áudio. A comunicação é feita através de SIP

\subsection{Vídeo}

\subsection{Gerência de conferência}

\section{Conclusão}~\cite{roesler_iva}

\todo{trabalhos futuros: captura de vídeo, múltiplos vídeos, módulo de apresentação, bloco de notas}
%
% The following two commands are all you need in the
% initial runs of your .tex file to
% produce the bibliography for the citations in your paper.
\bibliographystyle{abbrv}
\bibliography{paper}  % sigproc.bib is the name of the Bibliography in this case
% You must have a proper ".bib" file
%  and remember to run:
% latex bibtex latex latex
% to resolve all references
%
% ACM needs 'a single self-contained file'!
%
\balancecolumns
% That's all folks!
\end{document}
